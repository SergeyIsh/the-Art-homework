\documentclass[12pt]{article}
\usepackage[utf8]{inputenc}
\usepackage[T1]{fontenc}
\usepackage{amsmath,amsfonts,amssymb}
\usepackage{graphicx}
\usepackage{a4wide}
\title{Comparative analysis of my projects}
%\author{not specified}
\date{}
\begin{document}
\maketitle

\section{Jokes apart: machine learning and explainable artificial intelligence for humor generation}
LLMs are still not so good at humor generation. This project will focus specifically on the humor slice and try to find a good metric of quality for it.
\begin{enumerate}
\item \emph{The impact:} the project helps to improve the quality of LLMs in generation humor texts.
\item \emph{The consistency:} the main challenge is how to construct a metric to estimate the quality of humor.
\item \emph{The novelty:} Generally, it is not a very well-researched topic, and I can focus on the Russian language.
\item \emph{My contribution:} provide a metric to estimate the quality of humoristic text.
\item \emph{The project focuses:} the quality of humoristic texts can be estimated and improved.
\end{enumerate}

\section{Spot the bot: semantic paths of natural language texts}
The main idea is that the natural language texts have some specific patterns that allow to classify texts as LLM-generated and human-generated.
\begin{enumerate}
\item \emph{The impact:} the project allows to detect cheating in writing assignments and estimate whether the internet page was AI-generated or not.
\item \emph{The consistency:} the practical solution of LLM-generated text includes an implementation of a special watermark algorithm.
\item \emph{The novelty:} most of the researches focus on English corpus of text, I can focus on Russian texts.
\item \emph{My contribution:} realisation of watermark algorithm on Russian corpus of texts.
\item \emph{The project focuses:} LLM-generated texts still have some patterns that allow us to detect them.
\end{enumerate}

\section{LLM as universal interface}
The idea is to train LLM how to choose and use some basic classification model based on the text description of the problem.
\begin{enumerate}
\item \emph{The impact:} using ML models still requires some basic skills of using Python, even for the fit-predict interface. LLM can potentially provide an interface for users on how to solve simple classification tasks without using Python.
\item \emph{The consistency:} this is mostly about business, but generally the problem of transfering text description into model parameters is still uncovered and can be potentially solved in a more scientific way than just good propmt choosing.
\item \emph{The novelty:} I can't find any good example of such an interface.
\item \emph{My contribution:} a simple implementation of this interface, based on tg-bot, for instance.
\item \emph{The project focuses:} a text prompt is enough to describe a simple ML model.
\end{enumerate}

\section{Resume}
The project \emph{Jokes apart: machine learning and explainable artificial intelligence for humor generation} has the highest priority since. It is fun and looks interesting for me. I believe that the problem can be solved and spent a reasonable time. I also have some experience creating metrics for LLM models and believe that this experience will help me with this project.

\end{document}

